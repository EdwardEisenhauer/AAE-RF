\documentclass{article}
\usepackage{graphicx}
\usepackage{hyperref}
\usepackage{siunitx}
\usepackage{todonotes}

\title{RF Circuits Project Report}
\author{Sergiusz Warga 230757}
\date{2026-01-18}

\begin{document}

\maketitle

\section{Introduction}

This report presents the design and measurements of the microstrip high-pass filter.
The design was made in AWR software\footnote{\url{https://www.cadence.com/en_US/home/tools/system-analysis/rf-microwave-design.html}}.

\paragraph{Project parameters}

\begin{table}
  \centering
  \begin{tabular}{c|c}
    $Z_0$ & 50 Ohm \\
    $f_0$ & 1.5 GHz \\
    Window & Chebyshev \\
    Ripples & 0.5 dB
  \end{tabular}
  \caption{Microstrip high-pass filter goal parameters}
  \label{tab:project_parameters}
\end{table}

\paragraph{Project goals}

The optimisation task is to achieve the \qty{3}{\dB} loss at $f_0$ and the sharpest slope possible within the PCB constraints.
The circuit should be analysed up to \qty{3}{\GHz}.

\paragraph{Laminate parameters}

\begin{table}
  \centering
  \begin{tabular}{c|c}
    Dielectric constant (Er) & 4.5 \\
    Dielectric thickness (H)& 1.6 mm \\
    Conductor material & copper \\
    Conductor thickness (t) & 35 um \\
    Loss tangent (TAN) & 0.02
  \end{tabular}
  \caption{Laminate parameters}
  \label{tab:laminate_parameters}
\end{table}

\newpage
\section{Microstrip filter design}
% Start -> AWR
% Select all license features
% Wizards -> iFilter Filter Synthesis -> Design
The goal was to design a microstrip high-pass filter with a Chebyshev window with the use of shunt stubs with quarter wavelength equal stubs with the ripples below \qty{0.5}{\dB}.

For the initial filter design I used iFilter Filter Synthesis.
I entered the ripple value of \qty{0.5}{\dB}.
In the \textit{Specifications} I chose to start with the filter of degree 5.
I set $F_p$ to \qty{1.5}{\GHz} and $R_s$ and $R_L$ to \qty{50}{\ohm}.
\todo[inline]{What should I enter in the ElecLng[deg] field?}

In the Design Options I set the Technology's Microstrip to the values from \autoref{tab:laminate_parameters}.

I copy the values:

% Physical Model
% Dimensions: 106.213mm x 27.8844mm
% Area: 2961.68mm
% Layout grid size: 5mm
% Technology: Microstrip (H=1.6mm, Er=4.5, T=0.035mm, Hu=0mm)
% Parts
%   TLN: W=2.9972mm, L=13.638mm
%   SST: W=3.9809mm, L=12.960mm
%   STEP: W1=2.9972mm, W2=5.2437mm, Offset=0mm
%   TLN: W=5.2437mm, L=14.750mm
%   SST: W=3.9782mm, L=12.962mm
%   STEP: W1=5.2437mm, W2=5.0860mm, Offset=0mm
%   TLN: W=5.0860mm, L=14.772mm
%   SST: W=3.9771mm, L=12.962mm
%   TLN: W=5.0851mm, L=14.772mm
%   SST: W=3.9768mm, L=12.963mm
%   STEP: W1=5.0851mm, W2=5.2409mm, Offset=0mm
%   TLN: W=5.2409mm, L=14.750mm
%   SST: W=3.9768mm, L=12.963mm
%   STEP: W1=5.2409mm, W2=2.9972mm, Offset=0mm
%   TLN: W=2.9972mm, L=13.638mm

We got it.
I create New Schematic in Circuit Schematics.

I picked the MSUB Substrate, populated it with the values from \autoref{tab:laminate_parameters}.
I opened Tools -> TXLine to get the normalized conductivity.
I divided gold's 4.1 by copper's 5.88 to get $R_{rho}=0.697$

\missingfigure{Circuit schematic from AWR}

\missingfigure{Wideband graph for linear analysis of the circuit}

%wideband means maximum frequency on x axis to be twice the F0 
%*narrowband means a symmetrical frequency range around F0 that allow for more detailed analysis of the interesting region
%all the graphs should indicate (on graph or in their description) the interesting data points for analysis
%the graphs output in AWR can be easily exported to text file by selecting MENU Graph -> Export Trace Data (graph needs to be active on screen)

\newpage
\section{EM structure analysis}

derived from circuit schematic

\missingfigure{picture of the created EM structure with the 2D mesh on it}

\missingfigure{wideband graph for analysis of the EM structure}

\newpage
\section{Analysis of the physical board}

\missingfigure{picture of the created layout for fabricating 
the PCB (from the selected by yourself PCB designing software)}

\missingfigure{photo of the fabricated PCB and wideband graph 
for measurements of the PCB}

\newpage
\section{Conclusions}

\missingfigure{Wideband and narrowband combined graphs}

\todo[inline]{Write comments/conclusions}

\end{document}
